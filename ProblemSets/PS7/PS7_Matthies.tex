
\documentclass[12pt]{article}
 
\usepackage[margin=1in]{geometry} 
\usepackage{amsmath,amsthm,amssymb}
\usepackage{graphicx}
\newenvironment{statement}[2][Statement]{\begin{trivlist}
\item[\hskip \labelsep {\bfseries #1}\hskip \labelsep {\bfseries #2.}]}{\end{trivlist}}

\begin{document}
 
% --------------------------------------------------------------
%
%                         Start here
%
% --------------------------------------------------------------
 
\title{Problem Set 7} % replace with the problem you are writing up
\author{Adam Matthies} % replace with your name
\maketitle


\begin{statement}

560 logwages were missing. I think the logwage variable is MCAR. The listwise deletion of Beta one
varies quite differenly from the other three models. The predicitive model is the same to the mean
imputation but it is very dependent on the inputted values of x--which I used the mean values of
each column for the values of x. The mean imputation I would think would be the most accurate
because it is a large dataset so the values will tend toward the expected value. I got a Beta one
of .059 from mice. I am still confused on the theoretical part of multiple imputations, so I can't
expand any further on that. I was not able to include the multiple imputations as apart of the
combined modelsummary because I was having issues with my R code, but the modelsummary for just
the mice can be seen after running my R code.

\end{statement}

\begin{statement}

I am still working on finding the best data to do what I would like. I would like to create
some predictive model or gradient-based optimization model on some sort of financial/economic
data. Objective functions that intrigue me to optimize would be risk, solvency, profit, or something of the like.
\end{statement}
 

\begin{table}[ht]
\centering
\begin{tabular}{rllllll}
  \hline
 &    logwage &      hgc &   college &     tenure &      age &   married \\ 
  \hline
X & Min.   :0.0049   & Min.   : 0.0   & Length:2229        & Min.   : 0.000   & Min.   :34.00   & Length:2229        \\ 
  X.1 & 1st Qu.:1.3623   & 1st Qu.:12.0   & Class :character   & 1st Qu.: 1.583   & 1st Qu.:36.00   & Class :character   \\ 
  X.2 & Median :1.6551   & Median :12.0   & Mode  :character   & Median : 3.750   & Median :39.00   & Mode  :character   \\ 
  X.3 & Mean   :1.6252   & Mean   :13.1   &  & Mean   : 5.971   & Mean   :39.15   &  \\ 
  X.4 & 3rd Qu.:1.9362   & 3rd Qu.:15.0   &  & 3rd Qu.: 9.333   & 3rd Qu.:42.00   &  \\ 
  X.5 & Max.   :2.2615   & Max.   :18.0   &  & Max.   :25.917   & Max.   :46.00   &  \\ 
  X.6 & NA's   :560   &  &  &  &  &  \\ 
   \hline
\end{tabular}
\end{table}
 

\begin{document}

% Content on the current page


\newpage % This command starts a new page

\begin{table}[htbp]
\centering
\begin{tabular}{lcccc}
\toprule
 & MCARlm & MeanImputelm & PredictedValuelm \\
\midrule
(Intercept) & 0.627 & 0.808 & 0.826 \\
 & (0.109) & (0.087) & (0.088) \\
hgc & 0.062 & 0.049 & 0.046 \\
 & (0.005) & (0.004) & (0.004) \\
collegenot college grad & 0.146 & 0.160 & 0.168 \\
 & (0.035) & (0.026) & (0.026) \\
tenure & 0.023 & 0.015 & 0.014 \\
 & (0.002) & (0.001) & (0.001) \\
$I(age^{2})$ & 0.000 & 0.000 & 0.000 \\
 & (0.000) & (0.000) & (0.000) \\
marriedsingle & -0.024 & -0.029 & -0.032 \\
 & (0.018) & (0.014) & (0.014) \\
\midrule
Num.Obs. & 1,669 & 2,229 & 2,229 \\
R$^{2}$ & 0.195 & 0.132 & 0.112 \\
R$^{2}$ Adj. & 0.192 & 0.130 & 0.110 \\
AIC & 1,206.1 & 1,129.3 & 1,195.5 \\
BIC & 1,244.0 & 1,169.2 & 1,235.4 \\
Log.Lik. & -596.041 & -557.636 & -590.727 \\
F & 80.512 & 67.504 & 55.926 \\
RMSE & 0.35 & 0.31 & 0.32 \\
\bottomrule
\end{tabular}
\caption{Comparative Model Summaries}
\label{tab:model_comparison}
\end{table}


\end{document}






% --------------------------------------------------------------
%     You don't have to mess with anything below this line.
% --------------------------------------------------------------
 
\end{document}