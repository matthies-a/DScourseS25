\documentclass[12pt,english]{article}
\usepackage{mathptmx}

\usepackage{color}
\usepackage[dvipsnames]{xcolor}
\definecolor{darkblue}{RGB}{0.,0.,139.}

\usepackage[top=1in, bottom=1in, left=1in, right=1in]{geometry}

\usepackage{amsmath}
\usepackage{amstext}
\usepackage{amssymb}
\usepackage{setspace}
\usepackage{lipsum}

\usepackage[authoryear]{natbib}
\usepackage{url}
\usepackage{booktabs}
\usepackage[flushleft]{threeparttable}
\usepackage{graphicx}
\usepackage[english]{babel}
\usepackage{pdflscape}
\usepackage[unicode=true,pdfusetitle,
 bookmarks=true,bookmarksnumbered=false,bookmarksopen=false,
 breaklinks=true,pdfborder={0 0 0},backref=false,
 colorlinks,citecolor=black,filecolor=black,
 linkcolor=black,urlcolor=black]
 {hyperref}
\usepackage[all]{hypcap} % Links point to top of image, builds on hyperref
\usepackage{breakurl}    % Allows urls to wrap, including hyperref

\linespread{2}

\begin{document}

\begin{singlespace}
\title{Forward Guidance and its Effects on Consumer Spending}
\end{singlespace}

\author{Adam Matthies\thanks{Department of Economics, University of Oklahoma.\
E-mail~address:~\href{mailto:student.name@ou.edu}{adam.matthies@ou.edu}}}

% \date{\today}
\date{Spring 2025}

\maketitle

\begin{abstract}
\begin{singlespace}
I have no good quantitative evidence at the moment, but a positive relationship or even a negative
relationship on consumer spending is likely to be found after running linear regressions on Fed
policy decisions. Main challenge at the moment is getting data stationary to find estimated
causal effects.
\end{singlespace}

\end{abstract}
\vfill{}


\pagebreak{}


\section{Introduction}\label{sec:intro}
The Fed loves to be very vague on policy actions, so I am just looking for estimated cause and
effects of their policy actions or their believed policy actions. 

\section{Literature Review}\label{sec:litreview}
Something that stoked my interest on this topic was the power change from Alan Greenspan to 
Ben Bernanke. In a book from Ben Bernanke called "The Courage to Act", he talks about how he wanted
to prioritize a forward guidance and be more informative to the general public on expected inflation
numbers. So now, I am doing further research on the effects of setting these expectations that the
Fed likes to deem "Forward Guidance".



\section{Data}\label{sec:data}
My data all currently comes from FRED, but I still want to add and test variables with a LASSO
regression. I also need to add a categorical variable to my dataset of whether there was no
rate change, increased rates, or decreased rates for that month of data. My data is time series.



\section{Empirical Methods}\label{sec:methods}
While my approach explores a number of different approaches, the primary empirical model can be depicted in the following equation:

\begin{equation}
\label{eq:1}
\item
    \[
    \begin{aligned}
    \hat{ConsumerSpending} = &\ \hat{\beta}_0 + \hat{\beta}_1 \text{1-yr. Exp Inflation} + \hat{\beta}_2 \text{LFPR} + \\
    &\ \hat{\beta}_3 \text{Population} + \hat{\beta}_4 \text{Monetary Base} + \\
    &\ \hat{\beta}_5 \text{Personal Savings Rate} + \epsilon
    \end{aligned}
    \]
\end{equation}



\section{Research Findings}\label{sec:results}
Still need to go through these steps to get accurate research findings:

\begin{itemize}
    \item Add categorical variable for policy actioins
    \item Get stationary data, multicollinearity is not currently an issue
    \item Tool around with some other variables and LASSO
\end{itemize}

\section{Conclusion}\label{sec:conclusion}
No conclusion at the moment, but it can be assumed from theory that Forward Guidance will have
statistical significance on consumer spending.

\vfill
\pagebreak{}
\begin{spacing}{1.0}
\bibliographystyle{jpe}
\bibliography{PS11_Matthies.bib}
\addcontentsline{toc}{section}{References}
\end{spacing}

\vfill
\pagebreak{}
\clearpage

%========================================
% FIGURES AND TABLES 
%========================================
\section*{Figures and Tables}\label{sec:figTables}
\addcontentsline{toc}{section}{Figures and Tables}
%----------------------------------------
% Figure 1
%----------------------------------------
\begin{figure}[ht]
\centering
\bigskip{}
\includegraphics[width=.9\linewidth]{fig1.eps}
\caption{Figure caption goes here}
\label{fig:fig1}
\end{figure}

%----------------------------------------
% Table 1
%----------------------------------------

\caption{Summary Statistics of Variables of Interest}
\begin{table}[ht]
\centering
\begin{tabular}{rrrrr}
  \hline
 & Estimate & Std. Error & t value & Pr($>$$|$t$|$) \\ 
  \hline
(Intercept) & -134.5969 & 153.9046 & -0.87 & 0.3837 \\ 
  MonetaryBase & 0.0054 & 0.0015 & 3.59 & 0.0005 \\ 
  OneYearExpInfl & 3.1303 & 1.4459 & 2.17 & 0.0325 \\ 
  LFPR & 8.0501 & 2.7742 & 2.90 & 0.0045 \\ 
  Population & -0.0009 & 0.0002 & -4.54 & 0.0000 \\ 
  SavingsRate & 0.8082 & 0.3108 & 2.60 & 0.0106 \\ 
   \hline
\end{tabular}
\end{table}




\end{document}
